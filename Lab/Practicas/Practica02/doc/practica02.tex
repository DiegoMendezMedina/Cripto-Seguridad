\documentclass{article}
\usepackage[margin=1in]{geometry} 
\usepackage{amsmath,amsthm,amssymb,amsfonts, fancyhdr, color, comment, graphicx, environ}
\usepackage[dvipsnames]{xcolor}
\usepackage{amsmath}
\usepackage{mathrsfs}
\usepackage{listings} 
\usepackage{float}
\usepackage{graphicx}
\graphicspath{ {img/} }
\usepackage{mdframed}
\usepackage[shortlabels]{enumitem}
\usepackage{indentfirst}
\usepackage{hyperref}
\hypersetup{
    colorlinks=true,
    linkcolor=blue,
    filecolor=magenta,      
    urlcolor=blue,
}

\pagestyle{fancy}

\newenvironment{problem}[2][Pregunta]
    { \begin{mdframed}[backgroundcolor=gray!20] \textbf{#1 #2} \\}
    {  \end{mdframed}}

\newenvironment{solution}
    {\textbf{Respuesta:\\}}
    {}

\renewcommand{\qed}{\quad\qedsymbol}

\binoppenalty=\maxdimen
\relpenalty=\maxdimen

\lhead{Criptografía y Seguridad}
\rhead{420004358 Méndez Medina Diego\\420003708 Hernandez Uriostegui David} 
\chead{\textbf{Práctica 01}}

\begin{document}

\tableofcontents

\section{Cuestionario}

\begin{problem}{1}
  ¿Qué significa que un pentesting sea de Caja Blanca?
\end{problem}

\begin{problem}{2}
  Crea un diagrama explicando como funciana el bloque de RAM de una computador.
  Donde está el kernel, stack, el heap, etc.
\end{problem}

\begin{problem}{3}
  Menciona el significado de los siguientes registro:

  \begin{itemize}
  \item EAX
  \item EBX
  \item EXC
  \item EDX
  \item ESI
  \item EDI
  \item EBP
  \item ESP
  \item EIP
  \end{itemize}
\end{problem}

\begin{problem}{4}
  ¿Qué es little endian y big endian?¿Cuál usa tu procesador?
\end{problem}

\begin{problem}{5}
  ¿Qué es un segmento y que es un offset?
\end{problem}

\begin{problem}{6}
  ¿Que arquitectura tiene tu computadora?
\end{problem}

\begin{problem}{7}
  ¿Crées que esta vulnerabilidad desapareció con
  las nuevas funciones seguras como fgets?
\end{problem}
\newpage
%% Reporte 
\section{Reporte}

\subsection{Recopilación}

\begin{problem}{}
  En esta fase analizarás tus herramientas, el código y todo
  lo que cuentas para llevar a cabo la práctica.
\end{problem}


\subsection{Análisis}

\begin{problem}
  {} Identificarás las vulnerabilidades y como aprovecharlas. Documentaras
  todo lo que llegues a encontrar.
\end{problem}


\subsection{Explotación}

\begin{problem}
  {} En esta etapa documentarás la creación de tu exploit, como llegaste al puntero
  base, y como usaste las herramientas para lograrlo.
\end{problem}

\subsection{Post-explotación}

\begin{problem}
  {} En esta fase explicarás tu exploit resultante, también las conclusiones
  a las cuales llegaste. Explicarás como corregir el código de la práctica y anexarás la
  correción en su respectiva carpeta.
\end{problem}

\subsection{Imagina}

\begin{problem}{}
  En esta fase intenta imaginar todo lo que podrias hacer utilizando esta vulnerabilidad.
  Explicalo detalladamente y reflexiona el alcance de esta vulnerabilidad.
\end{problem}


\begin{thebibliography}{2} % 100 is a rangyesdom guess of the total number of
%references
\bibitem{} A
\bibitem{} B
\end{thebibliography}
%%%%%%%%%%%%% end %%%%%

\end{document}

