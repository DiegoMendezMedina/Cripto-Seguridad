\documentclass{article}
\usepackage[margin=1in]{geometry} 
\usepackage{amsmath,amsthm,amssymb,amsfonts, fancyhdr, color, comment, graphicx, environ}
\usepackage[dvipsnames]{xcolor}
\usepackage{amsmath}
\usepackage{mathrsfs}
\usepackage{listings} 
\usepackage{float}
\usepackage{graphicx}
\graphicspath{ {Imagenes/} }
\usepackage{mdframed}
\usepackage[shortlabels]{enumitem}
\usepackage{indentfirst}
\usepackage{hyperref}
\hypersetup{
    colorlinks=true,
    linkcolor=blue,
    filecolor=magenta,      
    urlcolor=blue,
}

\pagestyle{fancy}

\newenvironment{problem}[2][Pregunta]
    { \begin{mdframed}[backgroundcolor=gray!20] \textbf{#1 #2} \\}
    {  \end{mdframed}}

\newenvironment{solution}
    {\textbf{Respuesta:\\}}
    {}

\renewcommand{\qed}{\quad\qedsymbol}

\binoppenalty=\maxdimen
\relpenalty=\maxdimen

\lhead{Criptografía y Seguridad}
\rhead{420004358 Méndez Medina Diego\\420003708 Hernandez Uriostegui David} 
\chead{\textbf{Práctica 01}}

\begin{document}

\tableofcontents

\section{Cuestionario}

\begin{problem}
    {1} Menciona los tipos de protocolos de red
\end{problem}

\begin{solution}
    \begin{itemize}
        \item \textbf{Protocolos de comunicación} \\
        Son los que se encargan de establecer y gestionar las conexiones entre diferentes dispositivos.
        \begin{itemize}
            \item TCP/IP
            \item FTP
            \item SSH
            \item HTTP
        \end{itemize}

        \item \textbf{Protocolos de transporte}\\
        Son los encargados de transportar datos entre dispositivos.
        \begin{itemize}
            \item TCP
            \item UDP
        \end{itemize}

        \item \textbf{Protocolos de aplicación}\\
        Son aquellos que se utilizan para aplicaciones específicas, como el uso de redes sociales y el envio de correos electrónicos.
        \begin{itemize}
            \item SMTP
            \item POP3
            \item IMAP
            \item DNS
        \end{itemize}

        \item \textbf{Protocolos de enrutamiento}\\
        Aquellos utilizados para dirigir los paquetes de datos a través de la red.
        \begin{itemize}
            \item OSPF
            \item RIP
            \item BGP
        \end{itemize}

        \item \textbf{Protocolos de seguridad}\\
        Son los encargados de proteger la información transmitida en la red.
        \begin{itemize}
            \item SSL/TLS
            \item IPsec
            \item SSH
        \end{itemize}

        \item \textbf{Protocolos de gestión}\\
        Utilizados para administrar y monitorear dispositivos de red.
        \begin{itemize}
            \item SNMP
            \item NetFlow
        \end{itemize}
    \end{itemize}
\end{solution}

\begin{problem}
    {2} Respecto a tu pregunta anterior ¿Cómo funcionan? ¿Para qué sirven?
\end{problem}

\begin{solution}\\
Resuelto en la respuesta anterior.
\end{solution}  

\begin{problem}
    {3} ¿Qué es un sniffer?
\end{problem}

\begin{solution}
Se puede defininir como un software el cual está diseñado específicamente para redes, con el proposito de capturar y analizar los paquetes que se envían y reciben.\\
El \textit{sniffer} sería capáz de detectar qué estamos visitando, qué información enviamos, etc. De esta forma nuestra privacidad podría verse comprometida.
\end{solution}

\begin{problem}
    {4} OSINT, ¿Qué es? ¿Para qué sirve?
\end{problem}
\begin{solution}
\textit{Open Source INTelligence}\\

Es el conjunto de técnicas y herramientas para recopilar información pública, analizar los datos y correlacionarlos convirtiéndolos en conocimiento útil.\\

Su utlidad se basa en conseguir toda la información disponible en cualquier fuente pública sobre una empresa, persona física o cualquier otra cosa sobre la que queremos recopilar información para realizar una investigación, y haciendo que todo el conjunto de datos se convierta en inteligencia que nos sirva para ser más eficaces a la hora de querer obtener un resultado.
\end{solution}
\begin{problem}
    {5} Investiga los 5 OSINT más usados.
\end{problem}

\begin{solution}
    \begin{enumerate}
        \item \textbf{\textit{Google}}
        \item \textbf{\textit{Redes sociales}}
        \begin{itemize}
            \item Facebook
            \item Twitter
            \item Linkedln
            \item Instagram
        \end{itemize}
        \item \textbf{\textit{Wayback Machine}}\\
         Es un servicio y una base de datos que contiene copias de una gran cantidad de páginas de Internet, con esto se pueden consultsr diferentes versiones de páginas de Internet.

         \item \textbf{\textit{Maltego}}\\
         Es una herramienta de visualización de datos que permite establecer relaciones y conexiones entre personas, organizaciones y otras entidades.

         \item \textbf{\textit{Shodan}}\\
         Es un motor de busqueda que registra información sobre dispositivos conectados a Internet, como servidores, routers, cámaras web, etc.
    \end{enumerate}
\end{solution}

\begin{problem}
    {6} ¿Por qué el eslabón más débil de seguridad es el humano?
\end{problem}

\begin{solution}
    Esto se debe al hecho de que el ser humano no es perfecto y es facilmente influenciable.//

    Teniendo como consencuencias que seamos propensos a cometer errores y ser manipulados y engañados. Lo cual, en materia de seguridad, puede llegar a tener repercuciones como caer en estafas o extersiones.\\
    Dando un ejemplo más a fin, tomemos como ejemplo el phising o cuando compartimos contraseñas de redes sociales, cuentas de banco o del trabajo a personas sin pensar en las consecuencias que esto podría llegar a tener. Y esto más que nada es dado a que los humanos somos seres sentimentales y nuevamente, muy influenciables de lo que sucede en nuestro entorno.
    
\end{solution}

\begin{problem}
    {7} ¿Qué acciones haces para protegerte de ciberataques? 
\end{problem}

\begin{solution}
    Actualmente, ambos integrantes del equipo no estabamos al corriente en todos las posibles maneras en las que nuestra seguridad infórmatica puede llegar a ser perturbada. Por lo que en estos mommentos creemos que la acción más "segura" que usamos es la autenticación por 2 pasos.
\end{solution}

\begin{problem}
    {8}  ¿Crees que tus métodos preventivos son suficientes?
\end{problem}

\begin{solution}
    No, tal vez la autetenrticación por 2 pasos hoy en día es un buen mecanismo, pero creemos que podemos hacer más para proteger nuestra información.
\end{solution}


%% Reporte Script
\section{Reporte Script}

\subsection{Requisitos}

\begin{problem}
  {} ¿Qué problema queremos resolver o qué queremos saber? ¿Qué información
necesitamos? ¿Para qué?
\end{problem}

Nos interesa, mediante los protocolos con los que funciona internet y ciertas
herramientas ({\it software})
ya existentes, poder sacar la mayor información de como funcionan ciertos sistemas ({\it objetivos}), en especifíco recopilar como es que se comunican en internet o que servicios/procesos están
en constante ejecución, ya sea que cierto sistema operativo o algúna aplicación tenga falla
y tratar de identificar en que sentido el sistema es vulnerable.

\subsection{Indentificación de fuentes de información?}

\begin{problem}
  {} ¿Qué fuentes nos pueden aportar información
  valiosa y veraz?  
\end{problem}

Creemos que lo valioso depende de las habilidades y conocimiento de uno, sí sabemos
que en cierto puerto se ejecuta algún proceso con alguna vulnerabilidad entonces podriamos
atacar, o si por otro lado el protocolo que usa una red nos permite identificarnos como
un tercero. De donde nos podemos agarrar para recopilar es de las herramientas
ya creadas que permiten el paso de mensajes en el sistema web actual, así mismo tambíen las
personas pueden ser una fuente de información, ya sea que ellos la posean y la compartan o
tengan credenciales para algún sistema con las cuales se podría obtener.

Sabemos que las herramientas que permiten el paso de mensajes son veraces pues permiten un
correcto funcionamiento del internet, las personas podrian serlo o no, al igual que una
base de datos. En un principo la base no deberia tener error o información basura, pero con
gente capacitada quizá sea un cebo, es así que creemos que se puede confiar
en herramientas/software/información que es de dominio público y tienen algún funcionamiento
en la sociedad. Las vulnerabilidades que buscamos se auxilian de esta información, habilidades
y experiencia.

\subsection{Adquisición}

\begin{problem}
{} Etapa de obtención de la información. Explicación script.  
\end{problem}

Para el {\it script}, nos ayudamos en los comandos de linux compartidos por Ximena.


\subsection{Procesamiento}

\begin{problem}
{} Dar formato a toda la información “en bruto” obtenida en la anterior fase.
En caso de querer obtener más información explicar el posible uso de la misma, así como su
relevancia.  
\end{problem}



\subsection{Análisis}

\begin{problem}
  {}
  Generar datos de inteligencia a partir de todos los datos obtenidos, encontrando
relaciones entre estos que nos permitan llegar a conclusiones (vulnerabilidades).
\end{problem}

\subsection{Presentación}

\begin{problem}
  {}
  Darle a la información y conclusiones un formato en el que se pueda comprender
  de manera eficaz y sencilla presentando así tus propias propuestas de ataque.
\end{problem}

\begin{thebibliography}{2} % 100 is a rangyesdom guess of the total number of
%references
\bibitem{} https://autmix.com/blog/que-es-protocolo-red
\bibitem{} https://derechodelared.com/osint/
\end{thebibliography}
%%%%%%%%%%%%% end %%%%%

\end{document}

