\documentclass{article}
\usepackage[margin=1in]{geometry} 
\usepackage{amsmath,amsthm,amssymb,amsfonts, fancyhdr, color, comment, graphicx, environ}
\usepackage[dvipsnames]{xcolor}
\usepackage{amsmath}
\usepackage{mathrsfs}
\usepackage{listings} 
\usepackage{float}
\usepackage{graphicx}
\graphicspath{ {Imagenes/} }
\usepackage{mdframed}
\usepackage[shortlabels]{enumitem}
\usepackage{indentfirst}
\usepackage{hyperref}
\hypersetup{
    colorlinks=true,
    linkcolor=blue,
    filecolor=magenta,      
    urlcolor=blue,
}

\pagestyle{fancy}

\newenvironment{problem}[2][Pregunta]
    { \begin{mdframed}[backgroundcolor=gray!20] \textbf{#1 #2} \\}
    {  \end{mdframed}}

\newenvironment{solution}
    {\textbf{Respuesta:\\}}
    {}

\renewcommand{\qed}{\quad\qedsymbol}

\binoppenalty=\maxdimen
\relpenalty=\maxdimen

\lhead{Criptografía y Seguridad}
\rhead{} 
\chead{\textbf{Práctica 01}}

\begin{document}

\begin{section}{Cuestionario}  

\begin{problem}
    {1} Menciona los tipos de protocolos de red
\end{problem}

\begin{solution}
    \begin{itemize}
        \item \textbf{Protocolos de comunicación} \\
        Son los que se encargan de establecer y gestionar las conexiones entre diferentes dispositivos.
        \begin{itemize}
            \item TCP/IP
            \item FTP
            \item SSH
            \item HTTP
        \end{itemize}

        \item \textbf{Protocolos de transporte}\\
        Son los encargados de transportar datos entre dispositivos.
        \begin{itemize}
            \item TCP
            \item UDP
        \end{itemize}

        \item \textbf{Protocolos de aplicación}\\
        Son aquellos que se utilizan para aplicaciones específicas, como el uso de redes sociales y el envio de correos electrónicos.
        \begin{itemize}
            \item SMTP
            \item POP3
            \item IMAP
            \item DNS
        \end{itemize}

        \item \textbf{Protocolos de enrutamiento}\\
        Aquellos utilizados para dirigir los paquetes de datos a través de la red.
        \begin{itemize}
            \item OSPF
            \item RIP
            \item BGP
        \end{itemize}

        \item \textbf{Protocolos de seguridad}\\
        Son los encargados de proteger la información transmitida en la red.
        \begin{itemize}
            \item SSL/TLS
            \item IPsec
            \item SSH
        \end{itemize}

        \item \textbf{Protocolos de gestión}\\
        Utilizados para administrar y monitorear dispositivos de red.
        \begin{itemize}
            \item SNMP
            \item NetFlow
        \end{itemize}
    \end{itemize}
\end{solution}

\begin{problem}
    {2} Respecto a tu pregunta anterior ¿Cómo funcionan? ¿Para qué sirven?
\end{problem}

\begin{solution}\\
Respondido en la respuesta anterior.
\end{solution}  

\begin{problem}
    {3} ¿Qué es un sniffer?
\end{problem}

\begin{solution}
Se puede defininir como un software el cual está diseñado específicamente para redes, con el proposito de capturar y analizar los paquetes que se envían y reciben.\\
El \textit{sniffer} sería capáz de detectar qué estamos visitando, qué información enviamos, etc. De esta forma nuestra privacidad podría verse comprometida.
\end{solution}

\begin{problem}
    {4} OSINT, ¿Qué es? ¿Para qué sirve?
\end{problem}
\begin{solution}
\textit{Open Source INTelligence}\\

Es el conjunto de técnicas y herramientas para recopilar información pública, analizar los datos y correlacionarlos convirtiéndolos en conocimiento útil.\\

Su utlidad se basa en conseguir toda la información disponible en cualquier fuente pública sobre una empresa, persona física o cualquier otra cosa sobre la que queremos recopilar información para realizar una investigación, y haciendo que todo el conjunto de datos se convierta en inteligencia que nos sirva para ser más eficaces a la hora de querer obtener un resultado.
\end{solution}
\begin{problem}
    {5} Investiga los 5 OSINT más usados.
\end{problem}

\begin{solution}
    \begin{enumerate}
        \item \textbf{\textit{Google}}
        \item \textbf{\textit{Redes sociales}}
        \begin{itemize}
            \item Facebook
            \item Twitter
            \item Linkedln
            \item Instagram
        \end{itemize}
        \item \textbf{\textit{Wayback Machine}}\\
         Es un servicio y una base de datos que contiene copias de una gran cantidad de páginas de Internet, con esto se pueden consultsr diferentes versiones de páginas de Internet.

         \item \textbf{\textit{Maltego}}\\
         Es una herramienta de visualización de datos que permite establecer relaciones y conexiones entre personas, organizaciones y otras entidades.

         \item \textbf{\textit{Shodan}}\\
         Es un motor de busqueda que registra información sobre dispositivos conectados a Internet, como servidores, routers, cámaras web, etc.
    \end{enumerate}
\end{solution}

\begin{problem}
    {6} ¿Por qué el eslabón más débil de seguridad es el humano?
\end{problem}

\begin{solution}
    Esto se debe al hecho de que el ser humano no es perfecto y es facilmente influenciable.//

    Teniendo como consencuencias que seamos propensos a cometer errores y ser manipulados y engañados. Lo cual, en materia de seguridad, puede llegar a tener repercuciones como caer en estafas o extersiones.\\
    Dando un ejemplo más a fin, tomemos como ejemplo el phising o cuando compartimos contraseñas de redes sociales, cuentas de banco o del trabajo a personas sin pensar en las consecuencias que esto podría llegar a tener. Y esto más que nada es dado a que los humanos somos seres sentimentales y nuevamente, muy influenciables de lo que sucede en nuestro entorno.
    
\end{solution}

\begin{problem}
    {7} ¿Qué acciones haces para protegerte de ciberataques? 
\end{problem}

\begin{solution}
    Actualmente, ambos integrantes del equipo no estabamos al corriente en todos las posibles maneras en las que nuestra seguridad infórmatica puede llegar a ser perturbada. Por lo que en estos mommentos creemos que la acción más "segura" que usamos es la autenticación por 2 pasos.
\end{solution}

\begin{problem}
    {8}  ¿Crees que tus métodos preventivos son suficientes?
\end{problem}

\begin{solution}
    No, tal vez la autetenrticación por 2 pasos hoy en día es un buen mecanismo, pero creemos que podemos hacer más para proteger nuestra información.
\end{solution}

\end{section}
\begin{thebibliography}{2} % 100 is a random guess of the total number of
%references
\bibitem{} https://autmix.com/blog/que-es-protocolo-red
\bibitem{} https://derechodelared.com/osint/
\end{thebibliography}
%%%%%%%%%%%%% end %%%%%
\end{document}
