\documentclass{article}
\usepackage[margin=1in]{geometry} 
\usepackage{amsmath,amsthm,amssymb,amsfonts, fancyhdr, color, comment, graphicx, environ}
\usepackage[dvipsnames]{xcolor}
\usepackage{amsmath}
\usepackage{mathrsfs}
\usepackage{listings} 
\usepackage{float}
\usepackage{graphicx}
\graphicspath{ {img/} }
\usepackage{mdframed}
\usepackage[shortlabels]{enumitem}
\usepackage{indentfirst}
\usepackage{hyperref}
\hypersetup{
    colorlinks=true,
    linkcolor=blue,
    filecolor=magenta,      
    urlcolor=blue,
}

\pagestyle{fancy}

\newenvironment{problem}[2][Pregunta]
    { \begin{mdframed}[backgroundcolor=gray!20] \textbf{#1 #2} \\}
    {  \end{mdframed}}

\newenvironment{solution}
    {\textbf{Respuesta:\\}}
    {}

\renewcommand{\qed}{\quad\qedsymbol}

\binoppenalty=\maxdimen
\relpenalty=\maxdimen

\lhead{Criptografía y Seguridad}
\rhead{
    \texttt{420004358} \hspace{.9cm}  Méndez Medina Diego\\
    \texttt{420003708}  Hernandez Uriostegui David
} 
\chead{\textbf{Tarea 01}}

\begin{document}

%% David pares
%% Diego impares

%% 1 
\begin{problem}
  {1} Se tiene el siguiente mecanismo de cifrado para mensajes de una sola letra:

  \begin{align*}
    \mathcal{M} &= \{a,\,b,\,c,\,d,\,e,\,f,\,g,\,h,\,i,\,j,\,k,\,l,\,m,\,n,\,o,\,p,\,q,\,r,\,s,\,t,\,u,\,w,\,x,\,y,\,z\} \\
    K &= \{a,\,b,\,c,\,d,\,e,\,f,\,g,\,h,\,i,\,j,\,k,\,l,\,m,\,n,\,ñ,\,o,\,p,\,q,\,r,\,s,\,t,\,u,\,w,\,x,\,y,\,z\}\\
    E(k, m) &:= k+m\text{ mod } 26\\
    D(k, c) &:= c-k \text{ mod } 26
  \end{align*}


  \begin{enumerate}
  \item[a)] ¿ Es perfectamente seguro este mecanismo? Justifica.
  \item[b)] Si quitamos la llave ñ y lo demás se mantiene igual. ¿Cambia la seguridad
    del mecanismo? Justifica.
  \end{enumerate}
\end{problem}

{\bf Solución:}

\begin{enumerate}
\item[a)]

  No es perfectamente seguro pues no cumple la siguiente propiedad:

  ``Las cardinalidades de los conjuntos claves, mensajes y
  criptogramas son iguales''
  
  Mostraremos por que:
  
  Al tener una suma en la encriptación y una resta en la decriptación asumimos
  que:
  
  1. Operamos los índices de los caractéres tanto del mensaje plano como
  el de la contraseña (en su respectivo alfabeto).

  2. El mensaje cifrado es un número.

  Sin aún delimitar el alfabeto del criptograma(siguiente pregunta) ya sabemos
  que la cardinalidad de conjuntos claves es 27 pero de mensajes es 26.

  Por lo tanto no es perfectamente seguro.

  %% 1.b
\item[b)]
  
  Bajo este criterio la cardinalidad de conjuntos clave y mensaje sí son iguales,
  veremos que el de criptogramas no y que además no cumplen con la siguiente propiedad:

  ``Para toda pareja $(c, m)$, existe {\bf una} $k$ tal que $m$
  se cifra como $c$ usando $k$.''

  No importa si consideramos que los alfabetos estan en índice base cero o uno, llegaremos a
  la misma conclusión. Como en la operación usan módulo veintiseis consideramos
  pertinente usar índice base uno.

  Ahora tenemos como posibles claves los elementos del alfebeto utilizado en E.E.U.U.A
  (26 crácteres), así digamos para el mensaje 'a', tenemos los siguientes posibles valores
  comenzando con la llave 'a' y en orden cronologico hasta llegar a 'z':

  $$[2,3,4,5,6,7,8,9,10,11,12,13,14,15,16,17,18,19,20,21,22,23,24,25,26,27$$

  Para el cáracter 'z' los siguientes:

  $$[1,2,3,4,5,6,7,8,9,10,11,12,13,14,15,16,17,18,19,20,21,22,23,24,25,26]$$

  Observamos que es imposible para el mensaje 'a' producir el criptograma $1$, pues
  el índice de 'a' ya es 1 y con las claves operamos en un rango $\{1,..,26\}$.

  No existe una clave $k$ tal que 'a' se cifré como 1 usando $k$.

  Y aún más importante la longitud del conjunto de criptogramas($\{1,..,27\}$)
  no es igual a la de claves ni mensajes.

  {\bf *Nota:} Para conseguir los posibles valores utilizamos código escrito en
  Haskell que se encuentra en el siguiente
  \href{https://github.com/DiegoMendezMedina/Cripto-Seguridad/tree/main/Tareas/01/src/1.hs}{enlace}
\end{enumerate}

%% 2
\begin{problem}{2}
  Considera los mecanismos de cifrado adaptados a cadenas de bytes y descifra los siguientes
  criptotextos con la llave correspondiente.

  \begin{enumerate}
  \item[a)] Sustitución monoalfabética.
    
    C = {\tt 0xfb0762a891},

    llave de cifrado $k$ es la permutación $(04 \mapsto 91,\, 01 \mapsto 32,\, 75 \mapsto 62,\,
    \texttt{ff $\mapsto$ fb})$
    
    y los demás bytes quedan fijos. 
  \item[b)] Vigenére. $c = \texttt{0x00ab23cd45},\, k = \texttt{0xfa03}$.
  \item[c)] Afín. $c = \texttt{0x23aa7f},\, k = (255, 7)$.
  \end{enumerate}
\end{problem}

%% 3
\begin{problem}{3}
  El siguiente texto fue crifrado con una sustitución monoalfabética. Encuentra el texto claro en
  español y describe tu procedimiento de criptoanálisis. Puedes apoyarte del programa Ganzua
  \footnote{\href{https://ganzua.sourceforge.net/en/index.html}{Ganzüa}} u otro.

  \small
  %% Texto
  {\tt GKVQ JJQZH XQ FQVBKWH P XQNOSUVK SZR FRNBR QMBQZNKHZ HOQRZKOR OSPRN RWSRN BQZKRZ SZ OHJHV}
  
  {\tt BRZ LRVQOKXH R JR BKZBR ASQ GQ VQOHVXRVHZ JR XQNOVKLOKHZ ASQ YROQ QJ WQHWVRTH ZSUKH XQJ}
  
  {\tt GRVQ BQZQUVRVSG ZKZWSZR KGRWKZROKHZ YSGRZR LHXVKR OHZOQUKV LRZHVRGR GRN JRGQZBRUJQGQZBQ}
  
  {\tt XQNHJRXH R XQVQOYR Q KCASKQVXR P YRNBR XHZXQ LHXKR RJORZCRV JR GKVRXR NQ BQZXKRZ OHGH}
  
  {\tt GSVRJJRN XQJ GSZXH ORXQZRN XQ RORZBKJRXHN YHVVKUJQGQZBQ ZQWVHN P OHJWRZBQN OSPH JSWSUVQ}
  
  {\tt RNLQOBH FQKRNQ VQTHVCRXH LHV JR VQNROR ASQ VHGLKR OHZBVR QJJHN NS UJRZOR P JKFKXR OVQNBR}
  
  {\tt RSJJRZXH P VSWKQZXH QBQVZRGQZBQ HLSQNBR RJ LVHGHZBHVKH NHUVQ OSPR OKGR ZHN YRJJRURGHN}

  {\tt P R SZRN OKZOH H NQKN GKJJRN XQZBVH XQJ GRV RXFQVBKRNQ SZR LQASQÑR KNJR XQ RNLQOBH XQNQVBKOH}
  {\tt ASKCR NQR GRN RXQOSRXH XQOKV ASQ NS LHNKOKHZ NQ RXKFKZRUR WVROKRN R JRN NRJFRDQN VHGLKQZBQN}
  {\tt ASQ JR QZFHJFKRZ SZRN XHN GKJJRN GRN OQVOR RJCRURNQ HBVR KNJR GRN LQASQÑR YHVVKUJQGQZBQ}

  {\tt QNORVLRXR P QNBQVKJ VHXQRXR QZ FRVKRN LRVBQN LHV RGHZBHZRGKQZBHN XQ HNOSVRN VHORN}

  \normalsize

  Con frecuencias:

  \ttfamily
  R: 125, Q: 95, H: 66, N: 58, Z: 57, V: 56, K: 52, J: 43, O: 40, X: 39, S: 34, B: 33,
  G: 30, L: 18, U: 13, W: 11, F: 10, P: 10, A: 9, Y: 7, C: 5, T: 2, Ñ: 2, M: 1, D: 1

  \normalfamily
  %% Frecuencias
\end{problem}
%% 4
\begin{problem}{4}
  
\end{problem}

%% 5
\begin{problem}{5}
  Considera la siguiente modificación al sistema one-time real pad; se usan mensajes de longitud
  arbitraria pero acotada, es decir, $\mathcal{M} ) \{0,1\}^{\leq\mathcal{l}}$ (cadenas de bits de
  longitud entre 1 y $\mathcal{l}$). Oara encriptar se usan llaves de tamaño $\mathcal{l}$ y
  al aplicar el {\rm XOR} solo se usa el número de bits necesarios de la llave, es decir,
  $K = \{0,1\}^\mathcal{l}$ y $E(k,m):=k_{|m|} \oplus m$.

  \begin{enumerate}
  \item[a)] Muestra que este mecanismo no es perfectamente seguro.
    
  \item[b)] Construye un mecanismo perfectamente seguro para $\mathcal{M}$.
    Justifica la seguridad de tu respuesta.
  \end{enumerate}
\end{problem}

%% 6
\begin{problem}{6}
  
\end{problem}

%% 7
\begin{problem}{7}
  Sea $\Pi$ un mecanismo de cifrado simétrico arbitrario en el que $|K| < |\mathcal{M}|$.
  Considera el juego de distinguibilidad usando $\Pi$ y construye un adversario que gane
  el juego con probabilidad mayor a $\frac{1}{2}$.
\end{problem}

%% 8
\begin{problem}{8}
  
\end{problem}

%% 9
\begin{problem}{9}
  El archivo {\tt foto.cifrada} es una imagen encriptada con el mecanismo de César adaptado al
  alfabeto de bytes, es decir, el alfabeto es el conjunto $\{0,1,. . ., 255\}$ y los
  desplazamientos se hacen módulo 256. La imagen original está en formato PNG. Haz un programa
  para probar todas las llaves posibles y recupera la imagen. ¿Cuál es la llave? Anexa tu código.
\end{problem}

{\bf Solución:}

La llave es $180$, el codigo fue escrito en python y se encuentra en el siguiente
\href{https://github.com/DiegoMendezMedina/Cripto-Seguridad/tree/main/Tareas/01/src/9.py}{enlace}.

%% 10
\begin{problem}{10}
  
\end{problem}

\end{document}

